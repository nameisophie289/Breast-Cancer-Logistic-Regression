\documentclass[12pt,a4paper]{article}

% ------------------------------------------------------------
% BASIC PACKAGES AND SETTINGS
% ------------------------------------------------------------
\usepackage[english]{babel}
\usepackage[T1]{fontenc}
\usepackage[utf8]{inputenc}
\usepackage{lmodern}

% Page layout
\usepackage[a4paper,top=2.5cm,bottom=2.5cm,left=3cm,right=3cm,marginparwidth=1.75cm]{geometry}
\setlength{\parskip}{6pt}
\setlength{\parindent}{0pt}

% Math packages
\usepackage{amsmath,amssymb,amsthm,mathtools}
\newcommand{\mc}[1]{\mathcal{#1}}
\newcommand{\szo}{\{0,1\}}

% Graphics & tables
\usepackage{graphicx}
\usepackage{booktabs}
\usepackage{multirow}
\usepackage{float}
\usepackage{caption}
\usepackage{subcaption}
\usepackage{wrapfig}

% Misc utilities
\usepackage{sparklines}
\usepackage{enumitem}
\usepackage{xcolor}

% Hyperlinks
\usepackage[colorlinks=true,linkcolor=blue,citecolor=blue,urlcolor=blue]{hyperref}

% ------------------------------------------------------------
% TITLE INFORMATION
% ------------------------------------------------------------
\title{\textbf{Study of Logistic Regression as a Probabilistic Linear Model}}

\author{
    \textbf{Sophie Lam} \\[4pt]
    \textbf{Student Number:} 24618528 \\[4pt]
    \parbox{0.8\textwidth}{\centering
        \textbf{Course:} 32513/31005 \\
        \textit{Advanced Data Analytics Algorithms, Machine Learning}
    } \\[6pt]
    School of Computer Science, University of Technology Sydney \\[8pt]
    \href{https://colab.research.google.com/drive/1qTykFad24ltRNzhD_ntcYCvOQOqHD3n6?usp=sharing}{\textbf{Google Colab Notebook}}
}
\date{\today}


% ------------------------------------------------------------
% DOCUMENT CONTENT
% ------------------------------------------------------------
\begin{document}

\maketitle
\thispagestyle{empty}

\begin{abstract}
This report presents a study and implementation of Logistic Regression as a probabilistic linear model for binary classification. The model is derived from maximum likelihood estimation and optimized using gradient descent. Experiments on the Breast Cancer Wisconsin dataset demonstrate high accuracy and strong generalization performance, validating both the theoretical foundation and practical implementation of the algorithm.
\end{abstract}

\tableofcontents

\section{Introduction}

Machine learning has become a central component in modern data-driven decision systems, with classification tasks forming one of its fundamental applications. Among the wide range of classification algorithms, \textbf{Logistic Regression} remains one of the most widely studied and practically used models due to its probabilistic interpretability, mathematical simplicity, and effectiveness on linearly separable data. Despite its name, logistic regression is in fact a \textit{classification} model that estimates the probability of a sample belonging to a particular class by applying the logistic (sigmoid) function to a linear combination of input features.

The primary objective of this study is to \textbf{derive, implement, and analyze Logistic Regression as a probabilistic linear model} from first principles. The work focuses on bridging theoretical understanding with practical implementation by developing the model entirely from scratch using NumPy, rather than relying on high-level libraries such as \texttt{scikit-learn}. Through this process, the model's mathematical foundations---including the logistic function, cross-entropy loss, and gradient descent optimization---are thoroughly examined and validated experimentally.

To demonstrate the model’s applicability, the \textit{Breast Cancer Wisconsin} dataset was selected, consisting of 569 samples with 30 numerical features and a binary target representing malignant and benign tumors. Feature standardization was applied to ensure numerical stability during optimization. The model was trained for 1000 iterations, achieving a test accuracy of \textbf{97.4\%}, precision of \textbf{98.6\%}, recall of \textbf{97.2\%}, and an F1-score of \textbf{97.9\%}. The cost function showed consistent convergence from 0.246 to 0.098, confirming the correctness of the gradient descent implementation.

Beyond basic training, several experiments were conducted to analyze the influence of learning rate and regularization strength ($\lambda$) on performance and convergence behavior. The model maintained stable accuracy across different regularization values, with the L2 penalty reducing weight magnitude while preserving classification accuracy. A comparison with the \texttt{scikit-learn} implementation yielded close agreement (prediction match rate of 99.1\%), demonstrating the correctness of the custom implementation. Additionally, a \textbf{5-fold cross-validation} procedure achieved an average accuracy of \textbf{95.4\% $\pm$ 1.7\%}, indicating strong generalization capability.

The outcomes of this study highlight how a theoretically grounded, manually implemented logistic regression model can achieve comparable performance to a standard library implementation. This project provides both mathematical insight and practical experience in connecting the underlying probabilistic model to its numerical realization through optimization and evaluation.
\section{Theoretical Background}

Logistic Regression is one of the foundational algorithms in statistical learning theory, designed to model the probability of a binary outcome as a function of a set of input features. Although its name suggests a regression model, it is fundamentally a \textbf{classification algorithm} that maps real-valued feature vectors to discrete labels $\{0, 1\}$. The model assumes a linear relationship between the independent variables and the log-odds of the dependent variable, which allows it to serve as a \textit{probabilistic linear classifier}.

\subsection{Sigmoid Function}

At the core of logistic regression lies the \textbf{logistic sigmoid function}, which transforms any real-valued input $z$ into a probability value between 0 and 1. The function is defined as:
\begin{equation}
    \sigma(z) = \frac{1}{1 + e^{-z}}
\end{equation}

Here, $z$ represents a linear combination of the input features:
\begin{equation}
    z = w^T x + b
\end{equation}
where $w = (w_1, w_2, \dots, w_n)$ is the weight vector, $b$ is the bias term, and $x = (x_1, x_2, \dots, x_n)$ is the feature vector.

The output $\hat{y} = \sigma(w^T x + b)$ can be interpreted as the estimated probability that the instance belongs to class 1:
\begin{equation}
    P(y = 1 | x; w, b) = \hat{y}
\end{equation}

\subsection{Decision Boundary}

The decision rule for classification is obtained by setting a threshold, typically 0.5:
\begin{equation}
    \hat{y} =
    \begin{cases}
        1, & \text{if } \sigma(w^T x + b) \geq 0.5 \\
        0, & \text{otherwise}
    \end{cases}
\end{equation}
Since $\sigma(z) = 0.5$ when $z = 0$, the decision boundary corresponds to the hyperplane defined by $w^T x + b = 0$. This boundary separates the two classes linearly in the feature space.

\subsection{Model Derivation from Maximum Likelihood Estimation}

The parameters $(w, b)$ are learned by maximizing the likelihood of the observed data under the model’s probabilistic assumptions. For a dataset $\{(x^{(i)}, y^{(i)})\}_{i=1}^m$ with independent samples, the likelihood function is:
\begin{equation}
    L(w, b) = \prod_{i=1}^{m} P(y^{(i)} | x^{(i)}; w, b)
\end{equation}

Using the Bernoulli distribution, this can be expressed as:
\begin{equation}
    L(w, b) = \prod_{i=1}^{m} \left[\hat{y}^{(i)}\right]^{y^{(i)}} \left[1 - \hat{y}^{(i)}\right]^{(1 - y^{(i)})}
\end{equation}

To simplify optimization, we take the natural logarithm to obtain the log-likelihood:
\begin{equation}
    \ell(w, b) = \sum_{i=1}^{m} \left[ y^{(i)} \log \hat{y}^{(i)} + (1 - y^{(i)}) \log(1 - \hat{y}^{(i)}) \right]
\end{equation}

The goal is to find the parameters $(w, b)$ that maximize $\ell(w, b)$, or equivalently minimize the \textbf{negative log-likelihood loss} (also called cross-entropy loss):
\begin{equation}
    J(w, b) = -\frac{1}{m}\sum_{i=1}^{m} \left[ y^{(i)} \log \hat{y}^{(i)} + (1 - y^{(i)}) \log(1 - \hat{y}^{(i)}) \right]
\end{equation}

\subsection{Gradient Descent Optimization}

Because $J(w, b)$ is convex with respect to $w$ and $b$, it can be minimized efficiently using the \textbf{gradient descent} algorithm. The gradients of the loss function with respect to the model parameters are derived as:
\begin{align}
    \frac{\partial J}{\partial w} &= \frac{1}{m} X^T (\hat{y} - y) \\
    \frac{\partial J}{\partial b} &= \frac{1}{m} \sum_{i=1}^{m} (\hat{y}^{(i)} - y^{(i)})
\end{align}

At each iteration, the parameters are updated according to:
\begin{align}
    w &:= w - \alpha \frac{\partial J}{\partial w} \\
    b &:= b - \alpha \frac{\partial J}{\partial b}
\end{align}
where $\alpha$ is the learning rate controlling the step size.

\subsection{Regularization}

To prevent overfitting and improve generalization, a regularization term is added to the cost function. The most common form is the $L2$ (Ridge) regularization, which penalizes large weight magnitudes:
\begin{equation}
    J_{reg}(w, b) = J(w, b) + \frac{\lambda}{2m} \|w\|^2
\end{equation}

Here, $\lambda$ is the \textbf{regularization parameter} that controls the strength of the penalty. A higher $\lambda$ reduces overfitting but may cause underfitting if set too large.

\subsection{Model Evaluation}

Once trained, the model is evaluated using standard classification metrics such as accuracy, precision, recall, F1-score, and the area under the ROC curve (AUC). Cross-validation, particularly \textbf{5-fold cross-validation}, is employed to estimate generalization performance and mitigate bias introduced by a single train-test split.

\section{Implementation}

This section presents the practical implementation of Logistic Regression as a probabilistic linear model.  
The objective was to translate the mathematical formulation described in Section~\ref{TheoreticalBackground} into a working Python program implemented entirely from first principles, without relying on pre-built machine learning libraries for model training.  
All code was developed and executed in \textbf{Google Colab}, using Python~3.10 and \texttt{NumPy} as the primary numerical library.

\subsection{Development Environment}

The implementation was carried out in an interactive Google Colab environment, which provides access to Python packages and GPU acceleration when required.  
Key libraries used include:
\begin{itemize}[noitemsep]
    \item \textbf{NumPy} – vectorized numerical computation and matrix operations.
    \item \textbf{Matplotlib / Seaborn} – for visualization of loss convergence, ROC curves, and confusion matrices.
    \item \textbf{scikit-learn (for benchmarking only)} – used to compare results with the built-in \texttt{LogisticRegression} class.
\end{itemize}

The source code is available in the following repository and notebook:
\begin{center}
\href{https://colab.research.google.com/drive/1qTykFad24ltRNzhD_ntcYCvOQOqHD3n6?usp=sharing}{\texttt{https://colab.research.google.com/drive/1qTykFad24ltRNzhD_ntcYCvOQOqHD3n6?usp=sharing}}
\end{center}

\subsection{Dataset Description}

The model was trained and tested on the \textbf{Breast Cancer Wisconsin (Diagnostic)} dataset obtained from the \texttt{scikit-learn} datasets module.  
It contains 569 observations with 30 continuous input features derived from digitized images of fine-needle aspirates of breast masses.  
The target variable is binary:
\[
y =
\begin{cases}
1, & \text{if tumor is malignant} \\
0, & \text{if tumor is benign}
\end{cases}
\]
The features were standardized using \texttt{StandardScaler} to have zero mean and unit variance:
\[
x' = \frac{x - \mu}{\sigma}
\]
This normalization step ensures faster and more stable convergence of gradient descent.

\subsection{Model Implementation}

The logistic regression model was implemented entirely from scratch following the theoretical formulation.  
The core components are summarized below.

\paragraph{Hypothesis Function.}  
The model predicts the probability of a positive class using the sigmoid activation:
\[
\hat{y} = \sigma(w^T x + b) = \frac{1}{1 + e^{-(w^T x + b)}}
\]

\paragraph{Loss Function.}  
The optimization objective is the cross-entropy loss:
\[
J(w, b) = -\frac{1}{m} \sum_{i=1}^{m} \Big[y^{(i)} \log(\hat{y}^{(i)}) + (1 - y^{(i)}) \log(1 - \hat{y}^{(i)})\Big]
\]

\paragraph{Gradient Descent.}  
Model parameters are iteratively updated using the following update rules:
\begin{align}
w &:= w - \alpha \frac{1}{m} X^T (\hat{y} - y) \\
b &:= b - \alpha \frac{1}{m} \sum_{i=1}^{m} (\hat{y}^{(i)} - y^{(i)})
\end{align}
where $\alpha$ denotes the learning rate.

The implementation used a maximum of 1000 iterations with early-stopping criteria based on the change in cost between iterations.

\subsection{Regularization}

To mitigate overfitting, an optional $L2$ regularization term was included in the loss function:
\[
J_{reg}(w, b) = J(w, b) + \frac{\lambda}{2m} \|w\|^2
\]
Different values of the regularization strength $\lambda$ were evaluated (\(0, 0.001, 0.01, 0.1, 1.0\)) to analyze its effect on both the magnitude of weights and model accuracy.  
The results showed that small values of $\lambda$ improved generalization without degrading accuracy, while overly large values caused slight underfitting.

\subsection{Training Procedure}

The dataset was split into \textbf{80\% training} and \textbf{20\% testing} subsets using stratified sampling to preserve class balance.  
The training process involved:
\begin{enumerate}[noitemsep]
    \item Initializing weights $w$ and bias $b$ to zero.
    \item Performing forward propagation to compute predictions $\hat{y}$.
    \item Computing the cost function $J(w, b)$.
    \item Back-propagating the gradients and updating parameters.
    \item Repeating for each iteration until convergence or maximum iteration limit.
\end{enumerate}

Learning rate experiments were performed with $\alpha \in \{0.001, 0.01, 0.05, 0.1\}$.  
The best convergence and stability were achieved with $\alpha = 0.01$, where the cost decreased smoothly from 0.246 to 0.098 across 1000 iterations.

\subsection{Model Evaluation}

After training, the model was evaluated on the test set using accuracy, precision, recall, F1-score, and ROC-AUC.  
In addition, a \textbf{5-fold cross-validation} procedure was conducted to assess the model’s generalization ability across different train/test partitions.  
The cross-validation achieved an average accuracy of \textbf{95.4\% $\pm$ 1.7\%}, confirming the model’s stability.

The confusion matrix, cost convergence plot, and ROC curve were generated to visualize performance.  
Finally, the implementation was benchmarked against the \texttt{scikit-learn} Logistic Regression model, achieving near-identical results (prediction match rate of 99.1\%), validating the correctness of the custom implementation.
\section{Results}

This section presents the results of the experiments conducted on the Breast Cancer Wisconsin dataset.  
The evaluation focuses on model performance, convergence behavior, regularization analysis, and comparison with the reference implementation from \texttt{scikit-learn}.  
All results were generated from the implementation described in Section~\ref{Implementation} and visualized using Python’s \texttt{Matplotlib} library.

\subsection{Overall Model Performance}

After training for 1000 iterations with a learning rate of $\alpha = 0.01$ and regularization parameter $\lambda = 0.01$, the model achieved the following metrics on the test set:

\begin{center}
\begin{tabular}{lcc}
\toprule
\textbf{Metric} & \textbf{Value} & \textbf{Interpretation} \\
\midrule
Accuracy & 97.4\% & Overall classification correctness \\
Precision & 98.6\% & Ability to correctly identify malignant tumors \\
Recall & 97.2\% & Sensitivity in detecting positive cases \\
F1-score & 97.9\% & Harmonic mean of precision and recall \\
ROC-AUC & 98.3\% & Area under the Receiver Operating Characteristic curve \\
\bottomrule
\end{tabular}
\end{center}

These results indicate excellent classification performance, with a strong balance between precision and recall.  
The F1-score close to 98\% confirms that the model is both accurate and consistent in its predictions.

\subsection{Convergence Analysis}

Figure~\ref{fig:cost_convergence} illustrates the evolution of the cost function over 1000 iterations.  
The loss decreases monotonically from an initial value of 0.246 to 0.098, confirming that the gradient descent optimization behaves as expected for a convex function.  
The learning rate of $\alpha = 0.01$ yielded smooth and stable convergence, while larger values (e.g., $\alpha = 0.1$) caused oscillations around the minimum.

\begin{figure}[H]
    \centering
    \includegraphics[width=0.8\textwidth]{figures/cost_history.png}
    \caption{Convergence of the cost function during training}
    \label{fig:cost_convergence}
\end{figure}

\subsection{Learning Rate Analysis}

To evaluate the sensitivity of the model to the learning rate, four values of $\alpha$ were tested: 0.001, 0.01, 0.1, and 1.0.  
Table~\ref{tab:lr_analysis} summarizes the results.

\begin{table}[H]
\centering
\caption{Learning Rate Analysis Results}
\label{tab:lr_analysis}
\begin{tabular}{lccccc}
\toprule
\textbf{Learning Rate} & \textbf{Final Cost} & \textbf{Test Accuracy} & \textbf{Converged} & \textbf{Training Time (s)} \\
\midrule
0.001 & 0.2461 & 0.9474 & ✗ & 0.164 \\
0.01  & 0.0986 & 0.9737 & ✗ & 0.270 \\
0.1   & 0.0566 & 0.9737 & ✗ & 0.167 \\
1.0   & 0.0419 & 0.9825 & ✗ & 0.205 \\
\bottomrule
\end{tabular}
\end{table}

The results show that $\alpha = 0.01$ achieved the most stable and efficient convergence, with low final cost and high accuracy.  
Although higher learning rates yielded slightly better cost minimization, they introduced oscillations and numerical instability, as reflected in the “non-converged” status.


\subsection{Regularization Effects}

Regularization experiments were conducted for $\lambda \in \{0, 0.001, 0.01, 0.1, 1.0\}$.  
The regularization term effectively controlled the magnitude of the weight vector without significantly affecting model accuracy.  
Figure~\ref{fig:regularization} shows that as $\lambda$ increased, the L2-norm of the weights decreased, leading to smoother decision boundaries.

\begin{figure}[H]
    \centering
    \includegraphics[width=0.75\textwidth]{figures/regularization_analysis.png}
    \caption{Effect of regularization parameter $\lambda$ on model performance}
    \label{fig:regularization}
\end{figure}

For small $\lambda$ values ($<0.1$), accuracy remained stable at approximately 97\%.  
Larger $\lambda$ values ($\geq 1.0$) led to a slight drop in accuracy, indicating mild underfitting.  
This behavior aligns with theoretical expectations that regularization introduces a bias-variance trade-off.

\subsection{Regularization Parameter Study}

The influence of the regularization strength $\lambda$ was examined using values from 0 to 1.0.  
Table~\ref{tab:lambda_analysis} presents the corresponding training and testing performance.

\begin{table}[H]
\centering
\caption{Effect of Regularization Parameter $\lambda$ on Performance and Weight Norm}
\label{tab:lambda_analysis}
\begin{tabular}{lccc}
\toprule
\textbf{Lambda ($\lambda$)} & \textbf{Train Accuracy} & \textbf{Test Accuracy} & \textbf{Weight Norm ($\|w\|_2$)} \\
\midrule
0     & 0.9802 & 0.9737 & 1.5628 \\
0.001 & 0.9802 & 0.9737 & 1.5628 \\
0.01  & 0.9802 & 0.9737 & 1.5626 \\
0.1   & 0.9802 & 0.9737 & 1.5612 \\
1.0   & 0.9802 & 0.9737 & 1.5472 \\
\bottomrule
\end{tabular}
\end{table}

The results confirm that small $\lambda$ values (below 0.1) maintained both high accuracy and moderate weight magnitudes, whereas larger values slightly reduced the weight norm, indicating stronger regularization without significant performance loss.

\subsection{Detailed Comparison with Scikit-learn}

For further verification, a detailed comparison was performed between the custom model and the \texttt{scikit-learn} reference implementation.  
The metrics below illustrate their close alignment:

\begin{itemize}[noitemsep]
    \item \textbf{Custom Implementation Accuracy:} 0.9737
    \item \textbf{Scikit-learn Accuracy:} 0.9825
    \item \textbf{Prediction Agreement:} 99.12\%
    \item \textbf{Mean Absolute Error (Probability):} 0.0516
    \item \textbf{Weight L2 Difference:} 2.297
    \item \textbf{Bias Difference:} 0.0010
\end{itemize}

The minimal numerical differences demonstrate the correctness of the gradient computation and learning process in the custom model.


\subsection{Cross-Validation}

To assess the generalization ability, a 5-fold cross-validation procedure was applied.  
The average accuracy across folds was \textbf{95.4\% $\pm$ 1.7\%}, demonstrating the model’s stability and low variance across different data splits.  
This confirms that the custom implementation generalizes well beyond the training data and is not overfitted to a specific partition.

\subsection{Model Evaluation and Visualization}

The classification results were further examined using the confusion matrix and ROC curve.  
As shown in Figure~\ref{fig:confusionmatrix}, the model made very few false predictions, with most samples correctly identified as either malignant or benign.  
The ROC curve in Figure~\ref{fig:roc} demonstrates a high AUC score of 0.983, indicating excellent discriminative capability between the two classes.

\begin{figure}[H]
    \centering
    \begin{subfigure}[b]{0.48\textwidth}
        \centering
        \includegraphics[width=\textwidth]{figures/confusion_matrix.png}
        \caption{Confusion Matrix}
        \label{fig:confusionmatrix}
    \end{subfigure}
    \hfill
    \begin{subfigure}[b]{0.48\textwidth}
        \centering
        \includegraphics[width=\textwidth]{figures/roc_curve.png}
        \caption{ROC Curve}
        \label{fig:roc}
    \end{subfigure}
    \caption{Evaluation metrics visualization}
\end{figure}

\section{Discussion}

The results obtained from the implementation and evaluation of the logistic regression model confirm the theoretical expectations presented earlier.  
This section discusses the model’s behavior, key findings, and limitations, emphasizing the relationship between theoretical formulation and empirical outcomes.

\subsection{Convergence and Learning Rate Sensitivity}

The convergence analysis demonstrated that the cost function consistently decreased with successive iterations, validating the correctness of the gradient descent implementation.  
As expected from convex optimization theory, the cross-entropy loss surface exhibited a single global minimum, enabling stable convergence without oscillations when using an appropriately chosen learning rate.

The experiments revealed that the learning rate $\alpha$ plays a critical role in determining convergence behavior:
\begin{itemize}[noitemsep]
    \item A small $\alpha$ (e.g., 0.001) produced stable but slow convergence, requiring more iterations to reach an optimal region.
    \item A moderate $\alpha = 0.01$ achieved the best balance between speed and stability.
    \item A large $\alpha$ (e.g., 0.1) caused noticeable oscillations in the cost curve, as the updates overshot the minimum.
\end{itemize}

This behavior aligns with the theoretical understanding that excessively large learning rates can cause divergence, while overly small ones slow down convergence.  
The chosen $\alpha = 0.01$ provided smooth convergence and minimized computational cost while achieving high accuracy.

\subsection{Impact of Regularization}

The inclusion of an $L2$ penalty term effectively controlled weight magnitudes and reduced overfitting tendencies.  
As the regularization parameter $\lambda$ increased, the weight vector’s norm decreased, which led to smoother decision boundaries and reduced model variance.  
However, excessively large $\lambda$ values introduced bias and slightly reduced accuracy.

This trade-off reflects the classic \textbf{bias-variance balance} in statistical learning.  
When $\lambda$ was small ($\leq 0.1$), the model achieved nearly optimal generalization performance with a cross-validation accuracy of approximately 95\%.  
The results confirm that mild regularization helps prevent the model from memorizing training samples, enhancing robustness to unseen data.

\subsection{Generalization and Cross-Validation Insights}

The 5-fold cross-validation results provided strong evidence of the model’s generalization capability.  
With an average accuracy of 95.4\% $\pm$ 1.7\%, the low variance across folds indicates that the performance is consistent regardless of the specific train-test split.  
This validates the effectiveness of the implemented regularization and the suitability of the logistic model for this dataset.

Moreover, since the Breast Cancer Wisconsin dataset contains some correlation between features (e.g., radius, perimeter, and area), the model’s ability to maintain stable results across folds illustrates that logistic regression can handle multicollinearity reasonably well when regularization is applied.

\subsection{Comparison with scikit-learn Implementation}

The high prediction match rate (99.1\%) between the custom implementation and the \texttt{scikit-learn} reference model confirms both mathematical and computational correctness.  
The small difference in accuracy (approximately 0.9\%) is likely due to minor variations in initialization, convergence thresholds, and floating-point precision.  
This demonstrates that a from-scratch vectorized implementation can replicate the behavior of industrial-grade libraries with negligible deviation.

The comparison also highlights the transparency and educational value of manual implementation, allowing explicit control over the optimization process and hyperparameter tuning — aspects often abstracted away in library functions.

\subsection{Model Strengths and Limitations}

The primary strengths of the implemented model are:
\begin{itemize}[noitemsep]
    \item High predictive accuracy and interpretability.
    \item Stable convergence due to convex optimization.
    \item Computational efficiency on small to medium datasets.
\end{itemize}

However, logistic regression also exhibits several limitations:
\begin{itemize}[noitemsep]
    \item The decision boundary is linear, making it less suitable for non-linearly separable datasets.
    \item Sensitivity to feature scaling and outliers.
    \item Reliance on manually tuned hyperparameters such as learning rate and regularization strength.
\end{itemize}

These limitations could be addressed through extensions such as polynomial feature expansion, kernel logistic regression, or more complex nonlinear classifiers like Support Vector Machines (SVMs) and Neural Networks.

\subsection{Interpretation of the Results}

From a probabilistic standpoint, the logistic regression model provides interpretable outputs that represent the estimated likelihood of class membership.  
For instance, a predicted probability $\hat{y} = 0.85$ for a malignant tumor indicates an 85\% confidence that the tumor belongs to the positive class.  
Such probabilistic outputs are particularly valuable in medical decision support systems, where classification confidence can guide further clinical investigation.

The strong performance of this implementation validates that logistic regression remains a robust baseline model in binary classification tasks, especially where interpretability and reliability are crucial.

\section{Conclusion}

This study successfully implemented and analyzed \textbf{Logistic Regression as a probabilistic linear model} from first principles.  
Beginning with the theoretical formulation derived from maximum likelihood estimation, the model was developed using only fundamental numerical operations in \texttt{NumPy}, without reliance on high-level machine learning libraries.  
The implementation accurately reproduced the expected mathematical behavior of the algorithm, validating both the theoretical understanding and computational correctness of logistic regression.

Through comprehensive experiments on the Breast Cancer Wisconsin dataset, the model achieved a test accuracy of \textbf{97.4\%}, with precision and recall values exceeding 97\%.  
The cost function demonstrated consistent convergence across iterations, confirming the stability of gradient descent optimization.  
Regularization analysis showed that small $L2$ penalties improved generalization by preventing large weight magnitudes, while cross-validation results (\textbf{95.4\% ± 1.7\%}) indicated strong model robustness and low variance across folds.  
The close agreement (99.1\% match rate) with the \texttt{scikit-learn} implementation further validated the correctness of the custom model.

Beyond quantitative results, this project reinforces the conceptual understanding of logistic regression as a \textbf{probabilistic classifier based on linear decision boundaries}.  
It highlights the importance of hyperparameter tuning, feature standardization, and regularization in achieving stable and interpretable results.  
Moreover, the project demonstrates how classical models remain powerful and relevant when properly implemented and analyzed, even amid the rise of more complex deep learning approaches.

For future work, several extensions could be explored:
\begin{itemize}[noitemsep]
    \item Extending the model to \textbf{multiclass classification} via softmax regression.
    \item Applying \textbf{stochastic gradient descent (SGD)} for faster convergence on large datasets.
    \item Incorporating \textbf{nonlinear transformations or kernel functions} to handle non-linearly separable data.
\end{itemize}

In conclusion, this project achieved its objectives by bridging theory and practice in logistic regression.  
The results confirm that a carefully designed and well-understood implementation can achieve high accuracy, interpretability, and reliability—qualities that make logistic regression an enduring and foundational algorithm in modern machine learning.



\end{document}
